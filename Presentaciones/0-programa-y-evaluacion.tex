\documentclass[handout,xcolor=dvipsnames]{beamer}

\usepackage[utf8]{inputenc}
\usepackage{default}
\usetheme[width=50pt]{ULatina}   % Bergen, Darmstadt
\usecolortheme[named=Green]{structure}
\usepackage{graphicx}
\usepackage{pgfpages}
\usepackage{tikz}
\usepackage[spanish]{babel}
\usepackage[normalem]{ulem} % to use \sout{} for strikethrough

%\setbeameroption{show notes on second screen}
%\setbeamersize{sidebar width left=50pt}


\title[BINGE-53]{BINGE-53 Sistemas Digitales II}
\subtitle{Programa del curso, evaluación y cronograma}
\author{Prof. Jorge Rivera~Guti\'errez}
\institute{Universidad Latina de Costa Rica\\ Ingenier\'\i a en Electr\'onica}
\logo{\includegraphics[height=45pt]{world.png}}
\date{II cuatrimestre 2014}
\newcommand{\pageframe}[1]{\frame{\begin{center}{ \Huge #1 }\end{center}}}

\begin{document}

\begin{frame}
 \maketitle
\end{frame}

\begin{frame}
 \begin{center}
  \Large Créditos: 3\\~\\
  Requisito: Sistemas Digitales I\\~\\
  Período: VI\\~\\
  Modalidad: cuatrimestral\\~\\
 \end{center}
\end{frame}

\section{Descripción del curso}

\begin{frame}{Descripción del curso}
Este curso amplia y ofrece una continuidad de los conocimientos adquiridos en el
curso de Sistemas Digitales I. Inicia con la realización de máquinas de estados
finitos; trata los dispositivos de lógica programable y de almacenamiento de
información y enfatiza en el diseño y programación con los mismos y concluye con
el estudio de los elementos de interface con los sistemas analógicos y una
introducción sobre el microprocesador.
\end{frame}

\section{Objetivos}

\pageframe{Objetivos}

\subsection{Objetivo general}

\begin{frame}{Objetivo general}
  \begin{block}{Objetivo general}
  \begin{itemize}
	\item   Ampliación y nivelación de los conocimientos sobre almacenamiento
  
	\item  Ampliar los conocimientos en la terminología, conceptos y utilización de circuitos digitales.
  
	\item Comprensión de la teoría digital dirigida al almacenamiento de datos.
  
	\item  Introducción a los conceptos básicos de microprocesadores.
  \end{itemize}
  \end{block}
\end{frame}

\subsection{Objetivos específicos}

\begin{frame}{Objetivos específicos}
  \begin{block}{Objetivos específicos}
    \begin{itemize}
      \item<1> Lograr el dominio sobre el almacenamiento de datos, sus operaciones fundamentales y técnicas.
      \item<1> Conocer y aplicar los conceptos sobre los tipos de memorias y su funcionamiento.
      \item<1> Conocer, usar y aplicar conceptos de memoria.
    \end{itemize}
  \end{block}
\end{frame}


\section{Contenidos}

\pageframe{Contenidos}

\subsection[Máquinas]{Máquinas Secuenciales de Estados Finitos}

\begin{frame}{Máquinas Secuenciales de Estados Finitos}
  \begin{block}{}
  \begin{itemize}
    \item Máquinas síncronas y asíncronas.
    \item Diagramas de estados.
    \item Máquinas secuenciales: Modelos de Mealy y Moore.
    \item Controladores de Richards.
    \item Diseño de sistemas de control con máquinas secuenciales.
  \end{itemize}
  \end{block}
\end{frame}

\subsection[PLD]{Dispositivos de Lógica Programable (PLD)}

\begin{frame}{Dispositivos de Lógica Programable (PLD)}
\begin{block}{}
\begin{itemize}
 \item Clasificación de los PLD según su arquitectura.
 \item Dispositivos PLD, SPLD, CPLD y FPGA
 \item Programación de dispositivos PLD.
 \item Introducción a los lenguajes de descripción de circuitos (HDL)
\end{itemize}
\end{block}
\end{frame}

\subsection[Almacenamiento]{Dispositivios de almacenamiento de información}

\begin{frame}{Dispositivios de almacenamiento de información}
\begin{block}{}
\begin{itemize}
 \item Arquitectura de una memoria.
 \item Clasificación de las memorias semiconductoras.
 \item Parámetros de fabricante.
 \item Memorias de solo lectura.
 \item Memorias de lectura y escritura.
 \item Diagramas temporales de memorias.
 \item Arreglos y bancos de memoria.
 \item Diseño de sistemas de decodificación de memoria.
 \item Distribución y tipos de memoria en una computadora personal.
\end{itemize}
\end{block}
\end{frame}

\subsection[Interface]{Interface con Sistemas Analógicos}

\begin{frame}{Interface con Sistemas Analógicos}
\begin{block}{}
\begin{itemize}
 \item Principio de conversión analógico-digital (A/D) y digital-analógico (D/A).
 \item Tipos de convertidores.
 \item Errores presentes en los convertidores.
 \item Convertidores D/A
 \item Convertidores A/D
 \item Simbología y parámetros de fabricante
\end{itemize}
\end{block}
\end{frame}

\subsection[Microprocesador]{Introducción sobre el microprocesador}

\begin{frame}{Introducción sobre el microprocesador}
\begin{block}{}
\begin{itemize}
 \item El concepto de computadora digital.
 \item Organización del sistema de computadora básica.
 \item Elementos básicos de una microcomputadora
 \item El microprocesador.
\end{itemize}
\end{block}
\end{frame}

\section{Metodología}

\pageframe{Metodología}

\begin{frame}{Metodología}

Se emplearán estrategias didácticas que favorezcan la construcción del conocimiento y el aprendizaje significativo. Se desarrollarán prácticas de carácter comprobatorio, como funcionamiento de proyectos de diseño que solucionen problemas propuestos por el docente.

\end{frame}

\section{Estrategias de aprendizaje}

\pageframe{Estrategias de aprendizaje}

\begin{frame}{Estrategias de aprendizaje} 

\begin{footnotesize}   Dentro de la estrategia metodológica que se plantea, se pretende al estudiante como protagonista y responsable de su proceso de aprendizaje a través de un proceso de planificación sistemática de los contenidos del curso en el que se entrelazan contenidos generales y específicos de ingeniería. Estos constituyen la  columna vertebral sobre la que se propiciarán los instrumentos teóricos propios de la disciplina con la que se proveen opciones para aprender por sí mismo, desarrollar la capacidad de pensar y la habilidad para la solución de problemas, construyendo colectivamente el conocimiento, así como desarrollar la capacidad de poder evaluar y tomar decisiones acerca de su propio proceso de aprendizaje desarrollando habilidades cognitivas y capacidades mediante la redacción de informes, la resolución de problemas, los proyectos individuales, el trabajo en equipo y estimulando las capacidades de comunicación de los educandos con una  profunda aplicación de las tecnologías de la información y las comunicaciones e Internet.
\end{footnotesize}
\end{frame}

  
\section{Evaluaci\'on}

\pageframe{Evaluación}

\begin{frame}{Evaluación}
\begin{center}
\begin{tabular}{|l|r|}\hline
	2 Exámenes parciales	&	40\%\\\hline
	1 Examen final		&	30\%\\\hline
	4 Proyectos de laboratorio &	20\%\\\hline
	Tareas y quices	&	10\%\\\hline\hline
	TOTAL			&	100\%\\\hline
\end{tabular}
\end{center}
\end{frame}

\section{Observaciones}

\pageframe{Observaciones}


\begin{frame}{Presentación de informes y evaluaciones}
\begin{block}{Presentación de informes y evaluaciones}
  \begin{itemize}[<+->]
    \item Los informes y evaluaciones deberán ser realizados con una ortografía y redacción de nivel universitario.
    \item Los informes deberán ser preparados usando el formato IEEE Transactions. La descripción de este formato y las plantillas para el mismo pueden ser encontrados en \url{http://bit.ly/bN3uDr}.
    \item Los informes \textbf{deberán} ser realizados utilizando el sistema de preparación de documentos \LaTeX.
    \item Las imágenes y los diagramas utilizados en los informes deberán ser de buena calidad y generados utilizando programas apropiados para dicha tarea.
  \end{itemize}
\end{block}
\end{frame}


\begin{frame}{Presentación de informes y evaluaciones}
\begin{block}{Presentación de informes y evaluaciones}
  \begin{itemize}[<+->]
    \item Todo documento preparado para este curso deberá ser entregado en el aula virtual en los periodos especificados. Para algunos informes se solicitará también una versión impresa.
    \item En los informes se asignará un 10\% correspondiente a la calidad del documento. Estos puntos podrán perderse por el incumplimiento de los requerimientos de formato, redacción, ortografía y calidad del documento.
  \end{itemize}
\end{block}
\end{frame}


\begin{frame}{Asistencia}
\begin{block}{Asistencia}
  \begin{itemize}[<+->]
  	\item El horario del curso es jueves de 17:00 a 19:20.
    \item El curso es de asistencia obligatoria. Se requiere de una asistencia al 80\% de las horas lectivas.
    \item El curso consta de 15 sesiones. Se requiere la asistencia a 12 sesiones para aprobar el curso.
    \item Las ausencias podrán ser justificadas por iniciativa del estudiante cuando haya una razón que así lo amerite.
  \end{itemize}
\end{block}
\end{frame}

\begin{frame}{Asistencia}
\begin{block}{Asistencia}
  \begin{itemize}[<+->]
	\item Las ausencias a evaluaciones deberán ser tramitadas de acuerdo con lo especificado en el reglamento de la Universidad.
    \item Las llegadas tardías después de las 17:45 o el retiro por más de 45 minutos de la clase serán consideradas como ausencias, a excepción de aquellos casos donde el estudiante haya concluido las asignaciones correspondientes a la sesión de ese día.
  \end{itemize}
\end{block}
\end{frame}

\section{Bibliografía}

\pageframe{Bibliografía}
\nocite{*}

\begin{frame}{Bibliografía}

\begin{thebibliography}{5}
 \bibitem{Texto}
  Monk, S. 
  \textit{Programming Arduino. Getting Started with Sketches}.
  Estados Unidos de América: McGraw-Hill Professional,
  2011.
 \bibitem{Cons1}
  Galeano, G.
  \textit{Programación de Sistemas Embebidos en C}.
  Mexico: Alfaomega,
  2009.
 \bibitem{Cons2}
  Torrente, O.
  \textit{Arduino. Curso Práctico de Formación}.
  McGraw Hill,
  Mexico: Alfaomega,
  2013.
 \bibitem{Cons3}
  Valdés, F. y Pallás, R.
  \textit{Microcontroladores. Fundamentos y Aplicaciones con PIC}.
  Mexico: Alfaomega,
  2007.
\end{thebibliography}

\end{frame}

\section{Cronograma}

\pageframe{Cronograma}

\begin{frame}{Cronograma}
 \begin{center}
 {\small
  \begin{tabular}{|c|c|c|}\hline
  
   Semana 	& \multicolumn{1}{c|}{Contenido Temático} & \multicolumn{1}{c|}{Actividades} \\ \hline \hline
   1 		& Principios 		&  Revisión del programa \\ \hline
   2 		& PICs: Generalidades & Charla + Expo 1 \\ \hline
   3 		& PICs: ASM			& Charla + Expo 1\\ \hline
   4 		& PICs: ASM + I/O  		& Charla + Expo 1 \\ \hline
   5		& PICs: ASM	+ I/O	& Charla + Práctica \\ \hline
   6 		& PICs: 	& Charla + Práctica \\ \hline

  
  \end{tabular}}
 \end{center}
\end{frame}

\begin{frame}{Cronograma}
 \begin{center}
 {\small
  \begin{tabular}{|c|c|c|}\hline
  
   Semana 	& \multicolumn{1}{c|}{Contenido Temático} & \multicolumn{1}{c|}{Actividades} \\ \hline \hline
   7 		& \multicolumn{2}{c|}{Prueba escrita} \\ \hline
   8 		& PIC: Interrupciones	& Charla + Discusión + Expo 2 \\ \hline
   9 		& PIC: Timers 		& Charla + Discusión + Expo 2\\ \hline
   10 		& PIC: 		& Charla + Discusión + Expo 2 \\ \hline
   11		& Arduino		& Charla + Discusión  \\ \hline
   12 		& Desarrollo de soluciones & Discusión + Resolución Prob \\ \hline
   13 		& Desarrollo de soluciones & Discusión + Resolución Prob \\ \hline
   14		& Desarrollo de soluciones & Discusión + Resolución Prob \\ \hline
   15 		& \multicolumn{2}{c|}{Prueba escrita}  \\ \hline
  \end{tabular}}
 \end{center}
\end{frame}



\end{document}
