\documentclass[letterpaper,12pt]{IEEEtran}
\usepackage[spanish]{babel}
\usepackage[utf8]{inputenc}
%\usepackage[margin=1in]{geometry}
\usepackage{graphicx}

\title{Práctica: Parcial 3}
\author{Prof. Jorge Rivera}

\begin{document}

\maketitle

\section*{Ejercicio 1}

Escriba un módulo en Verilog que realice el funcionamiento de una ALU de 4 bits, similar al 74381.

Implemente el módulo en Quartus. Asigne las entradas a pines conectados a switches y la(s) salida(s) en LEDs de una tarjeta DE1. Demuestre el funcionamiento.

\section*{Ejercicio 2}

Escriba un módulo en Verilog que funcione como un flip-flop tipo D, con clear (activo en bajo) y clock en el flanco positivo.

Escriba un módulo en Verilog que utilice múltiples instancias del flip-flop tipo D anterior para crear un registro de 8 bits entrada paralela, salida paralela, con reset, similar al 74273.

Implemente el módulo en Quartus. Asigne las entradas a pines conectados a switches y la(s) salida(s) en LEDs de una tarjeta DE1. Demuestre el funcionamiento.

\newpage



\section*{Ejercicio 3}

Escriba un módulo en Verilog que funcione como un flip-flop tipo RS con clock en el flanco positivo.

Escriba un módulo en Verilog que utilice múltiples instancias del flip-flop tipo RS anterior para crear un registro de corrimiento de 8 bits entrada serie, salida serie, con reset, similar al 7491. Asuma que solo se require una entrada de datos (A), un reloj (CLK) y la salida positiva (Q). 

Implemente el módulo en Quartus. Asigne las entradas a pines conectados a switches y la(s) salida(s) en LEDs de una tarjeta DE1. Demuestre el funcionamiento.

\section*{Ejercicio 4}

Escriba un módulo en Verilog que realice el funcionamiento de un sumador completo de un bit.

Utilice su módulo creado anteriormente para crear un sumador completo de seis bits con acarreo de salida.

Implemente el módulo en Quartus. Asigne las entradas a pines conectados a switches y la(s) salida(s) en LEDs de una tarjeta DE1. Demuestre el funcionamiento.

\end{document}
