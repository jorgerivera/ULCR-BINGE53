\documentclass[letterpaper,12pt]{IEEEtran}
\usepackage[spanish]{babel}
\usepackage[utf8]{inputenc}
%\usepackage[margin=1in]{geometry}
\usepackage{graphicx}

\title{Práctica: Parcial 2}
\author{Prof. Jorge Rivera}

\begin{document}

\maketitle

\section*{Ejercicio 1}

Escriba un módulo en Verilog que realice la operación lógica mostrada a continuación.

\[ L = A + ( D \oplus C )  \]

Implemente el módulo en Quartus. Asigne las entradas a pines conectados a switches y la(s) salida(s) en LEDs de una tarjeta DE1. Demuestre el funcionamiento.

\section*{Ejercicio 2}

Escriba un módulo en Verilog que realice la operación lógica mostrada a continuación.

\[ L = ( A + B ) \times ( D + \overline{C} )  \]

Implemente el módulo en Quartus. Asigne las entradas a pines conectados a switches y la(s) salida(s) en LEDs de una tarjeta DE1. Demuestre el funcionamiento.

\newpage



\section*{Ejercicio 3}

Escriba un módulo en Verilog que realice el funcionamiento de un multiplexor de 4 a 1, similar al 74135.

Implemente el módulo en Quartus. Asigne las entradas a pines conectados a switches y la(s) salida(s) en LEDs de una tarjeta DE1. Demuestre el funcionamiento.

\section*{Ejercicio 4}

Escriba un módulo en Verilog que realice el funcionamiento de un demultiplexor de 1 a 4, similar al 74139.

Implemente el módulo en Quartus. Asigne las entradas a pines conectados a switches y la(s) salida(s) en LEDs de una tarjeta DE1. Demuestre el funcionamiento.

\section*{Ejercicio 5}

Escriba un módulo en Verilog que realice el funcionamiento de un sumador completo de 2 bits, similar al 7482.

Implemente el módulo en Quartus. Asigne las entradas a pines conectados a switches y la(s) salida(s) en LEDs de una tarjeta DE1. Demuestre el funcionamiento.
\end{document}
